\chapter{WSDL Services}


\section{Using Web Services}

The \module{ZSI.wsdl} package provides a high-level client-side tool
for working with remote web services. The \class{ServiceProxy} class acts 
as a proxy for a service that has a WSDL description, and exposes methods 
that reflect the methods of the remote service. \class{ServiceProxy} objects 
are very straightforward - you initialize a proxy from a WSDL URL, then call 
methods on the proxy corresponding to the methods of the remote service.

\begin{verbatim}
from ZSI.wsdl import ServiceProxy

service = ServiceProxy('http://www.xmethods.net/sd/BabelFishService.wsdl')
value = service.BabelFish('en_de', 'This is a test!')
\end{verbatim}

The methods of \class{ServiceProxy} instances can be called with positional 
arguments (where the argument positions match the message descriptions in the 
associated WSDL) or keyword arguments (where the arguments match the message 
descriptions by name). Arguments to \class{ServiceProxy} methods must be  
compatible with the types required by the WSDL description.

The return value from a proxy method depends on the SOAP signature. If the 
remote service returns a single value, that value will be returned. If the 
remote service returns multiple ``out'' parameters, the return value of the 
proxy method will be a dictionary containing the out parameters indexed by 
name.  Complex return types are also supported, meaning an aggregation of
primitives can be returned from a service invocation.

\section{\module{ZSI.wsdl.ServiceProxy}}
\declaremodule{}{ZSI.wsdl.ServiceProxy}

The \module{ServiceProxy} module provides a convenient way to call 
remote web services that are described with WSDL. Service proxies 
expose methods that reflect the methods of the remote web service. 


\begin{classdesc}{ServiceProxy}{wsdl,\optional{, service\optional{, port}}}
The \class{ServiceProxy} class provides a high-level Pythonic way to 
call remote web services. A WSDL description must be available for the 
remote service.

The \var{wsdl} argument may be either the URL of the service description 
or an existing \class{WSDL} instance. The optional \var{service} and 
\var{port} name the service and port within the WSDL description that 
should be used. If not specified, the first defined service and port 
will be used.
\end{classdesc}


\subsection{ServiceProxy Objects}

A \class{ServiceProxy} instance, once instantiated, exposes callable 
methods that reflect the methods of the remote web service it 
represents. These methods may be called like normal methods, using 
*either* positional or keyword arguments (but not both).

When a method of a \class{ServiceProxy} is called with positional 
arguments, the arguments are mapped to the SOAP request based on 
the parameter order defined in the WSDL description. If keyword 
arguments are passed, the arguments will be mapped based on their 
names.


\subsection{ServiceProxy Example}

This example demonstrates instantiating a \class{ServiceProxy} object 
from a WSDL description to call a remote web service method.  It should be
noted that \class{ServiceProxy} supports the passing and receiving of complex
types.

\begin{verbatim}
from WebService.ServiceProxy import ServiceProxy

service = ServiceProxy('http://www.xmethods.net/sd/BabelFishService.wsdl')
value = service.BabelFish('en_de', 'This is a test!')
\end{verbatim}


\end{document}
