\newcommand{\XMLSchema}{\citetitle[]{XML Schema Specification}}
\newcommand{\WSDL}{\citetitle[http://www.w3.org/TR/wsdl]{Web Services 
Description Language }}
\newcommand{\SOAP}{\citetitle[http://www.w3.org/TR/soap]{SOAP 1.1 Specification}}
\newcommand{\WPY}{\program{wsdl2py }}
\newcommand{\WS}{\emph{Web Service}}
\newcommand{\URL}{\citetitle[]{Uniform Resource Locator}}
\newcommand{\XSD}{The XML Schema type library}
\newcommand{\WSI}{Basic Profile (WS-Interop)}
\newcommand{\DOCLIT}{\it{doc/literal}}
\newcommand{\RPCLIT}{\it{rpc/literal}}
\newcommand{\RPCENC}{\it{rpc/enc}}
\chapter{Introduction}

\ZSI{}, the Zolera SOAP Infrastructure, is a Python package that
provides an implementation of the SOAP specification, as described in
\SOAP.

This guide demonstrates how to use ZSI to develop \WS{} applications from a
\WSDL document.

This document is primarily concerned with demonstrating and documenting how to
use a \WS{} by creating and accessing Python data for the purposes of
sending and receiving SOAP messages.  Typecodes are used to marshall Python
datatypes into XML, which can be included in a SOAP Envelope.  The typecodes
are generated from information provided in the WSDL document, and generally
describe SOAP and XML Schema data types.  For a low-level treatment of
typecodes, and a description of SOAP-based processing see the ZSI manual.

\section{Acronyms and Terminology}

\begin{definitions}
\item{SOAP \newline Usually refering to the content and format of a message ultimately
sent and received by a \WS{}, see \SOAP{}}
\item{WSDL \newline A document describing a \WS{}'s interface, see
\WSDL{}}
\item{XMLSchema \newline Standard for modeling XML document
structure.  See \XMLSchema{}}
\item {schema document \newline A document containing a schema definition, all
of its components are defined in a single namespace.}
\item {schema \newline composed of one or more schema documents.}
\item{Element Declaration \newline A schema component that associates a
name with a type definition.  eg. \it{<element name="age" type="xsd:int">}, }
\item{GED \newline \it{Global Element Declaration}, an element declared at the
top-level of a schema.}
\item{ComplexType \newline A schema component type, normally specifies
attributes and children, means of aggregating information.}
\item{SimpleType \newline A simple data type like a string or integer.  The
\XMLSchema{} defines many built-in types. }
\item{\XSD \newline The \url{http://www.w3.org/2001/XMLSchema} namespace, which
contains definitions of various primitive types like string and integer, as well
as a compound type \it{complexType} used to create aggregate types.  
Conventionally the \emph{xsd} prefix is used to map to this schema.}
\item{\DOCLIT\newline document style with literal encoding}
\item{\RPCENC\newline rpc style with specified encoding, note not compatible
with \WSI}
\item{\RPCLIT\newline rpc style with literal encoding.}
\end{definitions}

\section{Overview}
This guide introduces the \WPY script, which generates python code representing
the various components defined in a WSDL document, and demonstrates how to 
create a client for a simple \WS from a WSDL document.

A \WS, in the context of this document, exposes a WSDL document describing the
service's interface, this document is typically available at a published URL (see
\URL).  The WSDL document defines SOAP bindings for communicating with the 
service. These bindings will be used to exchange SOAP messages, the contents of
which must adhere to the document structure specified by various schema. The 
schema used for the stated purpose are either included in the WSDL document, 
imported by it, or are built-in types (such as \emph{xsd}).  

\subsection{soap bindings}
There are two styles of SOAP bindings, \emph{rpc} and \emph{document}, and
different encoding mechanisms(see \WSDL). A \DOCLIT service is typically
described as an exchange of documents, while a \RPCENC or \RPCLIT service is
thought of in terms of remote procedure calls. Whether this distinction of
purpose is meaningful or useful is debatable.  \ZSI supports all three, but 
\RPCLIT and \DOCLIT are the focus of ongoing development (see \WSI).

\section{Not Covered}
\begin{enumerate}
 \item{How to create a WSDL document}
 \item {How to write XML Schema}
 \item {Interoperability}
 \item{How to use Web Services without WSDL}
\end{enumerate}
