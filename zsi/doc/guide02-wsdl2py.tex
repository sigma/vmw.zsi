\chapter{The wsdl2py Tool: WSDL/XMLSchema python code generator}

The second method uses wsdl2py.  Handling XML Schema
(see \citetitle[http://www.w3.org/XML/Schema]{XML Schema specification}) 
is one of the more difficult aspects
of using WSDL.  The class \class{WriteServiceModule}, which wsdl2py
uses, helps to hides these
details.  It generates a module with stub code for the client interface,
and a module that encapsulates the handling of XML Schema, automatically
generating type codes.

\section{wsdl2py}

\subsection{Command Line Flags}

\begin{description}
\item[-h, ---help] Display the help message and available command line
flags that can be passed to wsdl2py.
\item[-f FILE, ---file=FILE] Create bindings for the WSDL which is located at
the local file path.
\item[-u URL, ---url=URL] Create bindings for the remote WSDL which is located
at the provided URL.
\item[-x, ---schema] Just process a schema (xsd) file and generate the types
mapping file.
\item[-d, ---debug] Output verbose debugging messages during code generation.
\item[-a, ---address] WS-Addressing support.  The WS-Addressing schema must be
included in the corresponding WSDL.
\item[-b, ---complexType (more in subsection~\ref{subsection:complexType})]
Generate convenience functions for complexTypes.  This includes getters,
setters, factory methods, and properties.  ** Do NOT use with --simple-naming **
\item[-w, ---twisted] Generate a twisted.web client.  Dependencies: 
python\verb!>=!2.4, Twisted\verb!>=!2.0.0, TwistedWeb\verb!>=!0.5.0
\end{description}

NOTE: All the following options are considered to be unstable and/or not under
active development.
\begin{description}
\item[-e, ---extended] Do extended code generation.
\item[-z ANAME, ---aname=ANAME] Use a custom function, ANAME, for attribute name
creation.
\item[-t TYPES, ---types=TYPES] Dump the generated type mappings to a file
named, ``TYPES.py''.
\item[-s, ---simple-naming] Simplify the generated naming.
\item[-c CLIENTCLASSSUFFIX, ---clientClassSuffix=CLIENTCLASSSUFFIX] The suffic
to use for service client class. (default ``SOAP'')
\item[-m PYCLASSMAPMODULE, ---pyclassMapModule=PYCLASSMAPMODULE] Use the
existing existing type mapping file to determine the ``pyclass'' objects to be
used.  The module should contain an attribute, ``mapping'', which is a
dictionary of form, {schemaTypeName: (moduleName.py, className)}.
\end{description}

\subsection{The {\it complexType} option explained}
\label{subsection:complexType}

The {\it complexType} flag provides many conveniences to the programmer
pertaining to all aspects of dealing with complex types through the use of 
metaclasses. This option is the preferred way of generating client bindings and
types with the {\it wsdl2py} command.  The naming conventions used for all the
generated convenince functions make use of the typecode's {\it aname} attribute.
 By default wsdl2py appends an ``\verb!_!'' to the aname of a complex type and
 its elements.  For example, a complex type named ``\verb!employee!'' would have
 aname, ``\verb!_employee!''.

For the descriptions below we will use the following complex types for
reference.  The scenario assumes one is making a remote call to operation, {\it
WolframSearch}, which takes a request of element, {\it WolframSearch} as shown
below. {\small Note: {\it WolframSearchOptions} has been modified for this example.}
\begin{verbatim}
<xsd:complexType name='WolframSearchOptions'>
  <xsd:sequence>
    <xsd:element name='Query' minOccurs='0' maxOccurs='1' type='xsd:string'/>
    <xsd:element name='Limit' minOccurs='0' maxOccurs='1' type='xsd:int'/>
  </xsd:sequence>
  <xsd:attribute name='timeout' type='xsd:double' />
</xsd:complexType>
<xsd:element name='WolframSearch'>
  <xsd:complexType>
    <xsd:sequence>
      <xsd:element name='Options' minOccurs='0' maxOccurs='1' type='ns1:WolframSearchOptions'/>
    </xsd:sequence>
  </xsd:complexType>
</xsd:element>
\end{verbatim}

\begin{verbatim}
# Create a request object to operation WolframSearch
#   to be used as an example below
from WolframSearchService_services import *

port = WolframSearchServiceLocator().getWolframSearchmyPortType()
wsreq = WolframSearchRequest()
\end{verbatim}

\begin{description}
\item[Getters/Setters] A getter and setter function is defined for each element
of a complex type.  The functions are named \verb!get_element_ANAME! and
\verb!set_element_ANAME! respectively.  In this example, variable \var{wsreq}
has functions named \verb!get_element__Options! and \verb!set_element__Options!.
 In addition to elements, getters and setters are generated for the attributes
 of a complex type.  For attributes, just the name of the attribute is used in
 determining the method names, so get_attribute_NAME and set_attribute_NAME are
 created.

\item[Factory Methods] If an element of a complex type is a complex type itself,
then a conveniece factory method is created to get an instance of that types
holder class.  The factory method is named, \verb!newANAME!, so \var{wsreq} has
a factory method, \verb!new_Options!.

\item[Properties]
\citetitle[http://www.python.org/download/releases/2.2/descrintro/#property]{Python class properties}
are created for each element of the complex type.  They are mapped to the
corresponding getter and setter for that element.  To avoid name collisions the
properties are named, \verb!PNAME!, where the first letter of the type's pname
attribute is capitalized.  In our running example, \var{wsreq} has class
property, \verb!Options!, which calls functions \verb!get_element__Options! and
\verb!set_element__Options! under the hood.

\end{description}

\begin{verbatim}
# sample usage of the generated code

# get an instance of a Options holder class using factory method
opts = wsreq.new_Options()
wsreq.Options = opts

# assign values using the properties or methods
opts.Query = 'Newton'
opts.set_element__Limit(10)

# don't forget the attribute
opts.set_attribute_timeout(1.0)

# At this point the serialized wsreq object would resemble this:
# <WolframSearch>
#   <Options timeout="1.0" xsi:type="tns:WolframSearchOptions">
#     <Query xsi:type="xsd:string">Newton</Query>
#     <Limit xsi:type="xsd:double">10.0</Limit>
#   </Options>
# </WolframSearch>

# ready call the remote operation
wsresp = port.WolframSearch(wsreq)

# returned WolframSearchResponse type holder also has conveniences
print 'SearchTime:', wsresp.Result.SearchTime
\end{verbatim}

\section{ServiceProxy}

The \class{ServiceProxy} class provides calls to
web services. A WSDL description must be available for the 
service.  \class{ServiceProxy} uses \class{WSDLReader} internally to load 
a \class{WSDL} instance.

The user may build up a type codes module for use by \class{ServiceProxy}.

\begin{classdesc}{ServiceProxy}{wsdl,\optional{, service\optional{, port}}}

The \var{wsdl} argument may be either the URL of the service description 
or an existing \class{WSDL} instance. The optional \var{service} and 
\var{port} name the service and port within the WSDL description that 
should be used. If not specified, the first defined service and port 
will be used.

The following keyword arguments may be used:

\begin{tableiii}{l|c|p{30em}}{textrm}{Keyword}{Default}{Description}
\lineiii{\code{nsdict}}{\code{\{\}}}{Namespace dictionary to send in the
    SOAP \code{Envelope}}
\lineiii{\code{tracefile}}{\code{None}}{An object with a \code{write}
    method, where packet traces will be recorded.}
\end{tableiii}

A \class{ServiceProxy} instance, once instantiated, exposes callable 
methods that reflect the methods of the remote web service it 
represents. These methods may be called like normal methods, using 
*either* positional or keyword arguments (but not both).

The methods can be called with either positional or keyword arguments;
the argument types must be compatible with the types specified in the
WSDL description.

When a method of a \class{ServiceProxy} is called with positional 
arguments, the arguments are mapped to the SOAP request based on 
the parameter order defined in the WSDL description. If keyword 
arguments are passed, the arguments will be mapped based on their 
names.

\end{classdesc}

\subsection{Example}

The following example, using ServiceProxy,  shows a simple language
translation service that makes
use of the complex type structures defined in the module BabelTypes:

\begin{verbatim}
from ZSI import ServiceProxy
import BabelTypes

service = ServiceProxy('http://www.xmethods.net/sd/BabelFishService.wsdl',
		       typesmodule=BabelTypes)
value = service.BabelFish('en_de', 'This is a test!')
\end{verbatim}

The return value from a proxy method depends on the SOAP signature. If the 
remote service returns a single value, that value will be returned. If the 
remote service returns multiple ``out'' parameters, the return value of the 
proxy method will be a dictionary containing the out parameters indexed by 
name.  Because \class{ServiceProxy} makes use of the ZSI serialization / 
deserialization engine, complex return types are supported.  This means 
that an aggregation of primitives can be returned from or passed to a service
invocation according to any predefined hierarchical structure.


\section{Code Generation from WSDL and XML Schema}

This section covers wsdl2py, the second way ZSI provides to access WSDL
services.  Given the path to a WSDL service, two files are generated, a 
'service' file and a 'types' file, that one can then use to access the
service.  As an example, we will use the search service provided by Wolfram
Research Inc.\copyright{}, \url{http://webservices.wolfram.com/wolframsearch/}, 
which provides a service for searching the popular MathWorld site, 
\url{http://mathworld.wolfram.com/}, among others.

\begin{verbatim}
wsdl2py.py --complexType --url=http://webservices.wolfram.com/services/SearchServices/WolframSearch2.wsdl
\end{verbatim}

Run the above command to generate the service and type files.  wsdl2py uses
the {\it name} attribute of the {\it wsdl:service} element to name the resulting files.
In this example, the service name is {\it WolframSearchService}.  Therefore the files
{\it WolframSearchService_services.py} and {\it WolframSearchService_services_types.py}
should be generated.

The 'service' file contains locator, portType, and message classes.  
A locator instance is used to get an instance of a portType class, 
which is a remote proxy object. Message instances are sent and received 
through the methods of the portType instance.

The 'types' file contains class representations of the definitions and
declarations defined by all schema instances imported by the WSDL definition.
XML Schema attributes, wildcards, and derived types are not fully
handled.

\subsection{Example Use of Generated Code}

The following shows how to call a proxy method for {\it WolframSearch}.  It
assumes wsdl2py has already been run as shown in the section above.  The example
will be explained in greater detail below.

\begin{verbatim}
# import the generated class stubs
from WolframSearchService_services import *

# get a port proxy instance
loc = WolframSearchServiceLocator()
port = loc.getWolframSearchmyPortType()

# create a new request
req = WolframSearchRequest()
req.Options = req.new_Options()
req.Options.Query = 'newton'

# call the remote method
resp = port.WolframSearch(req)

# print results
print 'Search Time:', resp.Result.SearchTime
print 'Total Matches:', resp.Result.TotalMatches
for hit in resp.Result.Matches.Item:
    print '--', hit.Title
\end{verbatim}

Now each section of the code above will be explained.

\begin{verbatim}
from WolframSearchService_services import *
\end{verbatim}

We are primarily interested in the service locator that is imported.  The 
binding proxy and classes for all the messages are additionally imported.
Look at the {\it WolframSearchService_services.py} file for more information.

\begin{verbatim}
loc = WolframSearchServiceLocator()
port = loc.getWolframSearchmyPortType()
\end{verbatim}

Using an instance of the locator, we fetch an instance of the port proxy
which is used for invoking the remote methods provided by the service.  In
this case the default {\it location} specified in the {\it wsdlsoap:address}
element is used.  You can optionally pass a url to the port getter method to
specify an alternate location to be used.  The {\it portType} - {\it name} 
attribute is used to determine the method name to fetch a port proxy instance.
In this example, the port name is {\it WolframSearchmyPortType}, hence the 
method of the locator for fetching the proxy is {\it getWolframSearchmyPortType}.

The first step in calling {\it WolframSearch} is to create a request object
corresponding to the input message of the method.  In this case, the name of
the message is {\it WolframSearchRequest}.  A class representing this message
was imported from the service module.

\begin{verbatim}
req = WolframSearchRequest()
req.Options = req.new_Options()
req.Options.Query = 'newton'
\end{verbatim}

Once a request object is created we need to populate the instance with the
information we want to use in our request.  This is where the {\tt --complexType}
option we passed to wsdl2py will come in handy.  This caused the creation of 
functions for getting and setting elements and attributes of the type, class 
properties for each element, and convenience functions for creating new instances
of elements of complex types.  This functionality is explained in detail in 
subsection~\ref{subsection:complexType}.

Once the request instance is populated, calling the remote service is easy.  Using
the port proxy we call the method we are interested in.  An instance of the python
class representing the return type is returned by this call.  The \var{resp} object
can be used to introspect the result of the remote call.

\begin{verbatim}
resp = port.WolframSearch(req)
\end{verbatim}

Here we see that the response message, \var{resp}, represents type {\it WolframSearchReturn}.
This object has one element, {\it Result} which contains the search results for our
search of the keyword, {\tt newton}.

\begin{verbatim}
print 'Search Time:', resp.Result.SearchTime
...
\end{verbatim}

Refer to the wsdl for {\it WolframSearchService} for more details on the returned information.